%%%%%%%%%%%%%%%%%%%%%%%%%%%%%%%%%%%%%%%%
%           List of packages         %
%%%%%%%%%%%%%%%%%%%%%%%%%%%%%%%%%%%%%%%%


%% False text, just for demo
\usepackage{blindtext}
\usepackage{lipsum}

%%%%%%%%%%%%%%%%%%%%%%%%%%%%%%%%%%%%%%%%%%%%%%%%%%%%%%%%%%%%%%%%%%%%%

%% Font and typography settings    
\usepackage[utf8]{inputenc}							% LaTeX, comprend les accents !
\usepackage[T1]{fontenc}

\usepackage{amsmath}								% Allows to write mathematical equations
\newcommand{\RE}{\mathrm{Re}}
\newcommand{\IM}{\mathrm{Im}}
\usepackage{bbm}
\usepackage[framed,amsmath,thmmarks]{ntheorem}		% Allows to use theorems
\newtheorem{theorem}{Theorem}
\theoremstyle{definition}
\newtheorem{definition}{Definition}[section]
\usepackage[ruled]{algorithm2e}						% Allows to use algorithms
\usepackage{nicefrac}			% Allows to use 'inline' fractions 
\usepackage{commath}			% Allows to use the \abs math command
\usepackage{wasysym}            % Allows to use the \diametr command
\DeclareMathOperator{\Res}{Res}


%\usepackage{libertine,libertinust1math}				% Use Libertine ubuntu font for text and math see: https://tex.stackexchange.com/questions/59702/suggest-a-nice-font-family-for-my-basic-latex-template-text-and-math
		

%\usepackage{ae,aecompl}										% Utilisation des fontes vectorielles modernes
%\usepackage[upright]{fourier}
%\usepackage[]{utopia}

\usepackage{lmodern}
%% Maths                         
\usepackage{amsmath}		
\usepackage{amssymb}			% Allows to use mathematical symbols 
\usepackage{amsfonts}			% Allows to use mathematical fonts
%%%%%%%%%%%%%%%%%%%%%%%%%%%%%%%%%%%%%%%%%%%%%%%%%%%%%%%%%%%%%%%%%%%%%
%% Bibliography style

\usepackage[authoryear,square,semicolon,sort&compress,sectionbib]{natbib}		% Doit être chargé avant babel
\bibliographystyle{abbrvnat}%abbrvnat %plainnat $square

%\usepackage{chapterbib}
%	\renewcommand{\bibsection}{\section{Références}}		% Met les références biblio dans un \section (au lieu de \section*)
%%%%%%%%%%%%%%%%%%%%%%%%%%%%%%%%%%%%%%%%%%%%%%%%%%%%%%%%%%%%%%%%%%%%%
% Allure générale du document
\usepackage{enumerate}
\usepackage{enumitem}
\usepackage[section]{placeins}	% Place un FloatBarrier à chaque nouvelle section
\usepackage{epigraph}
\usepackage[%
    font={small,sf},
    labelfont=bf,
    format=hang,    
    format=plain,
    margin=0pt,
    width=0.8\textwidth,
]{caption}

\usepackage[nohints]{minitoc}		% Mini table des matières, en français
	\setcounter{minitocdepth}{2}	% Mini-toc détaillées (sections/sous-sections)
\usepackage[notbib]{tocbibind}		% Ajoute les Tables	des Matières/Figures/Tableaux à la table des matières

\usepackage{setspace}
\onehalfspacing

\usepackage{pgffor}
\setlength{\columnseprule}{0pt}
\setlength\columnsep{10pt}

\usepackage{emptypage}

\usepackage{afterpage}
\newcommand\blankpage{%
    \null
    \thispagestyle{empty}%
    \addtocounter{page}{-1}%
    \newpage}
    
%\usepackage{indentfirst}
%%%%%%%%%%%%%%%%%%%%%%%%%%%%%%%%%%%%%%%%%%%%%%%%%%%%%%%%%%%%%%%%%%%%%
%% Tables
\usepackage{multirow}
\usepackage{booktabs}
\usepackage{colortbl}
\usepackage{tabularx}
\usepackage{multirow}
\usepackage{threeparttable}
\usepackage{multicol}
\usepackage{etoolbox}
%	\appto\TPTnoteSettings{\footnotesize}
%\addto\captionsfrench{\def\tablename{{\textsc{Tableau}}}}	% Renome 'table' en 'tableau'

%%%%%%%%%%%%%%%%%%%%%%%%%%%%%%%%%%%%%%%%%%%%%%%%%%%%%%%%%%%%%%%%%%%%%
%% Eléments graphiques                    
\usepackage{xcolor}
\usepackage{graphicx}			% Permet l'inclusion d'images
\graphicspath{
  {figures/intro/}
  {figures/ch1/}
  {figures/ch2/}
  {figures/ch3/}
  {figures/ch4/}
  {figures/ch5/}
  {figures/ch6/}
  {figures/ch7/}
  {figures/ch8/}
  {figures/a1/}
  {figures/a2/}
  {figures/a3/}
  {figures/a4/}
  }

\usepackage[list=true]{subcaption}
\usepackage{pdfpages}
\usepackage{rotating}
\usepackage{pgfplots}
	\usepgfplotslibrary{groupplots}
\usepackage{eso-pic}
\usepackage{import}

%%%%%%%%%%%%%%%%%%%%%%%%%%%%%%%%%%%%%%%%%%%%%%%%%%%%%%%%%%%%%%%%%%%%%
%% Mise en forme du texte        
\usepackage{xspace}
%\usepackage[load-configurations = abbreviations]{siunitx}
%	\DeclareSIUnit{\MPa}{\mega\pascal}
%	\DeclareSIUnit{\micron}{\micro\meter}
%	\DeclareSIUnit{\tr}{tr}
%	\DeclareSIPostPower\totheM{m}
%	\sisetup{
%	locale = FR,
%	  inter-unit-separator=$\cdot$,
%	  range-phrase=~\`{a}~,     	% Utilise le tiret court pour dire "de... à"
%	  range-units=single,  		% Cache l'unité sur la première borne
%	  }

%\usepackage[version=3]{mhchem}	% Equations chimiques
\usepackage{textcomp}
\usepackage{array}
\usepackage{hyphenat}
\usepackage[absolute,overlay]{textpos}
\hyphenation{ex-am-ple hy-phen-a-tion short}
\hyphenation{long la-tex}
%%%%%%%%%%%%%%%%%%%%%%%%%%%%%%%%%%%%%%%%%%%%%%%%%%%%%%%%%%%%%%%%%%%%%
%% Hyperlinks - Navigation dans le document

\usepackage{hyperref}
\hypersetup{%
	pdfborder={0 0 0},
	plainpages=false,%
	pdfauthor={Author(s)},%
	pdftitle={Title},%
	pdfsubject={Subject},%
	bookmarksnumbered=true,%
	colorlinks=true,%
	citecolor=blue,%
	filecolor=blue,%
	linkcolor=blue,% you should probably change this to black before printing
	urlcolor=blue,%
	pdfstartview=FitH%
}
%%%%%%%%%%%%%%%%%%%%%%%%%%%%%%%%%%%%%%%%%%%%%%%%%%%%%%%%%%%%%%%%%%%%%
%% Packages qui doivent être chargés APRES hyperref	             
\usepackage[top=2.5cm, bottom=2cm, left=3cm, right=2.5cm,
			headheight=15pt]{geometry}

\usepackage{fancyhdr}
\setlength{\headheight}{15pt}

\pagestyle{fancy}
\renewcommand{\chaptermark}[1]{ \markboth{\thechapter.\ #1}{}}

\fancyhf{}
\fancyfoot[LE,RO]{\thepage} 
\fancyhead[LE]{\thechapter}
\fancyhead[LE]{\textsc{\leftmark}}
\fancyhead[RO]{\nouppercase{\rightmark}}
	
\usepackage[Sonny]{fncychap}

\makeatletter
\ChNameVar{\centering\Large\it}
\ChNumVar{\huge\it} 
\ChNameAsIs
%\ChTitleVar{\vspace*{-20pt} \centering\Huge\rm\bfseries}
\ChTitleVar{\centering\Huge\rm\bfseries}
\ChRuleWidth{1pt}

\patchcmd{\DOTI}{\vskip 40\p@}{\vskip 20\p@}{}{}
\patchcmd{\DOTIS}{\vskip 40\p@}{\vskip 20\p@}{}{}
\makeatother

%\usepackage[acronym,xindy,toc,numberedsection,ucmark]{glossaries}
%	\newglossary[nlg]{notation}{not}{ntn}{Notation} % Création d'un type de glossaire 'notation'
%	\makeglossaries
%	\loadglsentries{Glossaire}			% Utilisation d'un fichier externe pour la définition des entrées (Glossaire.tex)	

\usepackage{nomencl}
\makenomenclature

\usepackage{psl-cover}
\pslassetspath{figures/front_cover}
%%%%%%%%%%%%%%%%%%%%%%%%%%%%%%%%%%%%%%%%%%%%%%%%
\pdfcompresslevel0 %Accelerate the pdf compilation