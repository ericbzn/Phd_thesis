% created on 27-03-2020
% @author : ebazan
\documentclass[a4paper, twoside, 12pt]{book}
%%%%%%%%%%%%%%%%%%%%%%%%%%%%%%%%%%%%%%%%
%           List of packages         %
%%%%%%%%%%%%%%%%%%%%%%%%%%%%%%%%%%%%%%%%


%% False text, just for demo
\usepackage{blindtext}
\usepackage{lipsum}

%%%%%%%%%%%%%%%%%%%%%%%%%%%%%%%%%%%%%%%%%%%%%%%%%%%%%%%%%%%%%%%%%%%%%

%% Font and typography settings    
\usepackage[utf8]{inputenc}							% LaTeX, comprend les accents !
\usepackage[T1]{fontenc}

\usepackage{amsmath}								% Allows to write mathematical equations
\newcommand{\RE}{\mathrm{Re}}
\newcommand{\IM}{\mathrm{Im}}
\usepackage[framed,amsmath,thmmarks]{ntheorem}		% Allows to use theorems
\newtheorem{theorem}{Theorem}
\theoremstyle{definition}
\newtheorem{definition}{Definition}[section]
\usepackage[ruled]{algorithm2e}						% Allows to use algorithms
\usepackage{nicefrac}			% Allows to use 'inline' fractions 
\usepackage{commath}			% Allows to use the \abs math command
\usepackage{wasysym}            % Allows to use the \diametr command
\DeclareMathOperator{\Res}{Res}


%\usepackage{libertine,libertinust1math}				% Use Libertine ubuntu font for text and math see: https://tex.stackexchange.com/questions/59702/suggest-a-nice-font-family-for-my-basic-latex-template-text-and-math
		

%\usepackage{ae,aecompl}										% Utilisation des fontes vectorielles modernes
%\usepackage[upright]{fourier}
%\usepackage[]{utopia}

\usepackage{lmodern}
%% Maths                         
\usepackage{amsmath}		
\usepackage{amssymb}			% Allows to use mathematical symbols 
\usepackage{amsfonts}			% Allows to use mathematical fonts
%%%%%%%%%%%%%%%%%%%%%%%%%%%%%%%%%%%%%%%%%%%%%%%%%%%%%%%%%%%%%%%%%%%%%
%% Bibliography style

\usepackage[authoryear,square,semicolon,sort&compress,sectionbib]{natbib}		% Doit être chargé avant babel
\bibliographystyle{abbrvnat}%abbrvnat %plainnat $square

%\usepackage{chapterbib}
%	\renewcommand{\bibsection}{\section{Références}}		% Met les références biblio dans un \section (au lieu de \section*)
%%%%%%%%%%%%%%%%%%%%%%%%%%%%%%%%%%%%%%%%%%%%%%%%%%%%%%%%%%%%%%%%%%%%%
% Allure générale du document
\usepackage{enumerate}
\usepackage{enumitem}
\usepackage[section]{placeins}	% Place un FloatBarrier à chaque nouvelle section
\usepackage{epigraph}
\usepackage[%
    font={small,sf},
    labelfont=bf,
    format=hang,    
    format=plain,
    margin=0pt,
    width=0.8\textwidth,
]{caption}

\usepackage[nohints]{minitoc}		% Mini table des matières, en français
	\setcounter{minitocdepth}{2}	% Mini-toc détaillées (sections/sous-sections)
\usepackage[notbib]{tocbibind}		% Ajoute les Tables	des Matières/Figures/Tableaux à la table des matières

\usepackage{setspace}
\onehalfspacing

\usepackage{pgffor}
\setlength{\columnseprule}{0pt}
\setlength\columnsep{10pt}

\usepackage{emptypage}

\usepackage{afterpage}
\newcommand\blankpage{%
    \null
    \thispagestyle{empty}%
    \addtocounter{page}{-1}%
    \newpage}
    
%\usepackage{indentfirst}
%%%%%%%%%%%%%%%%%%%%%%%%%%%%%%%%%%%%%%%%%%%%%%%%%%%%%%%%%%%%%%%%%%%%%
%% Tables
\usepackage{multirow}
\usepackage{booktabs}
\usepackage{colortbl}
\usepackage{tabularx}
\usepackage{multirow}
\usepackage{threeparttable}
\usepackage{multicol}
\usepackage{etoolbox}
%	\appto\TPTnoteSettings{\footnotesize}
%\addto\captionsfrench{\def\tablename{{\textsc{Tableau}}}}	% Renome 'table' en 'tableau'

%%%%%%%%%%%%%%%%%%%%%%%%%%%%%%%%%%%%%%%%%%%%%%%%%%%%%%%%%%%%%%%%%%%%%
%% Eléments graphiques                    
\usepackage{xcolor}
\usepackage{graphicx}			% Permet l'inclusion d'images
\graphicspath{
  {figures/intro/}
  {figures/ch1/}
  {figures/ch2/}
  {figures/ch3/}
  {figures/ch4/}
  {figures/ch5/}
  {figures/ch6/}
  {figures/ch7/}
  {figures/ch8/}
  {figures/a1/}
  {figures/a2/}
  {figures/a3/}
  {figures/a4/}
  }

\usepackage[list=true]{subcaption}
\usepackage{pdfpages}
\usepackage{rotating}
\usepackage{pgfplots}
	\usepgfplotslibrary{groupplots}
\usepackage{eso-pic}
\usepackage{import}

%%%%%%%%%%%%%%%%%%%%%%%%%%%%%%%%%%%%%%%%%%%%%%%%%%%%%%%%%%%%%%%%%%%%%
%% Mise en forme du texte        
\usepackage{xspace}
%\usepackage[load-configurations = abbreviations]{siunitx}
%	\DeclareSIUnit{\MPa}{\mega\pascal}
%	\DeclareSIUnit{\micron}{\micro\meter}
%	\DeclareSIUnit{\tr}{tr}
%	\DeclareSIPostPower\totheM{m}
%	\sisetup{
%	locale = FR,
%	  inter-unit-separator=$\cdot$,
%	  range-phrase=~\`{a}~,     	% Utilise le tiret court pour dire "de... à"
%	  range-units=single,  		% Cache l'unité sur la première borne
%	  }

%\usepackage[version=3]{mhchem}	% Equations chimiques
\usepackage{textcomp}
\usepackage{array}
\usepackage{hyphenat}
\usepackage[absolute,overlay]{textpos}
\hyphenation{ex-am-ple hy-phen-a-tion short}
\hyphenation{long la-tex}
%%%%%%%%%%%%%%%%%%%%%%%%%%%%%%%%%%%%%%%%%%%%%%%%%%%%%%%%%%%%%%%%%%%%%
%% Hyperlinks - Navigation dans le document

\usepackage{hyperref}
\hypersetup{%
	pdfborder={0 0 0},
	plainpages=false,%
	pdfauthor={Author(s)},%
	pdftitle={Title},%
	pdfsubject={Subject},%
	bookmarksnumbered=true,%
	colorlinks=true,%
	citecolor=blue,%
	filecolor=blue,%
	linkcolor=blue,% you should probably change this to black before printing
	urlcolor=blue,%
	pdfstartview=FitH%
}
%%%%%%%%%%%%%%%%%%%%%%%%%%%%%%%%%%%%%%%%%%%%%%%%%%%%%%%%%%%%%%%%%%%%%
%% Packages qui doivent être chargés APRES hyperref	             
\usepackage[top=2.5cm, bottom=2cm, left=3cm, right=2.5cm,
			headheight=15pt]{geometry}

\usepackage{fancyhdr}
\setlength{\headheight}{15pt}

\pagestyle{fancy}
\renewcommand{\chaptermark}[1]{ \markboth{\thechapter.\ #1}{}}

\fancyhf{}
\fancyfoot[LE,RO]{\thepage} 
\fancyhead[LE]{\thechapter}
\fancyhead[LE]{\textsc{\leftmark}}
\fancyhead[RO]{\nouppercase{\rightmark}}
	
\usepackage[Sonny]{fncychap}

\makeatletter
\ChNameVar{\centering\Large\it}
\ChNumVar{\huge\it} 
\ChNameAsIs
%\ChTitleVar{\vspace*{-20pt} \centering\Huge\rm\bfseries}
\ChTitleVar{\centering\Huge\rm\bfseries}
\ChRuleWidth{1pt}

\patchcmd{\DOTI}{\vskip 40\p@}{\vskip 20\p@}{}{}
\patchcmd{\DOTIS}{\vskip 40\p@}{\vskip 20\p@}{}{}
\makeatother

%\usepackage[acronym,xindy,toc,numberedsection,ucmark]{glossaries}
%	\newglossary[nlg]{notation}{not}{ntn}{Notation} % Création d'un type de glossaire 'notation'
%	\makeglossaries
%	\loadglsentries{Glossaire}			% Utilisation d'un fichier externe pour la définition des entrées (Glossaire.tex)	

\usepackage{nomencl}
\makenomenclature

\usepackage{psl-cover}
\pslassetspath{figures/front_cover}
%%%%%%%%%%%%%%%%%%%%%%%%%%%%%%%%%%%%%%%%%%%%%%%%
\pdfcompresslevel0 %Accelerate the pdf compilation

% Thesis title
\newcommand{\PhDTitle}{Vision Methods for Navigation of Unmanned Aerial Vehicles (UAVs)} 

% Name
\newcommand{\PhDname}{Eric Bazán} 

\title{\PhDTitle}

\author{\PhDname}

\doctoralschool{Ingénierie des Systèmes, Matériaux, Mécanique, Énergétique - ISMME}{621}
\specialty{Morphologie Mathématique}
\date{JJ mois AAAA}

\jurymember{1}{Prénom NOM}{Titre, établissement}{Président}
\jurymember{2}{Prénom NOM}{Titre, établissement}{Rapporteur}
\jurymember{3}{Prénom NOM}{Titre, établissement}{Rapporteur}
\jurymember{4}{Prénom NOM}{Titre, établissement}{Examinateur}
\jurymember{5}{Prénom NOM}{Titre, établissement}{Examinateur}
\jurymember{6}{Prénom NOM}{Titre, établissement}{Examinateur}
\jurymember{7}{Petr Dokládal}{Titre, établissement}{Directeur de thèse}
\jurymember{8}{Eva Dokládalová}{Titre, établissement}{Directrice de thèse}

\frabstract{
  Ce travail de thèse traite de l'extraction de caractéristiques dans les images à partir de primitives de bas niveau pour la compréhension de scènes à partir de la détection et de la segmentation d'objets. Ce travail est divisé en trois parties.
  
  La première partie explore l'utilisation des contours d'image en combinaison avec certains concepts de la perception humaine tels que le principe de Helmholtz et les lois de la Gestalt. Ces méthodes sont ensuite appliquées à la tâche de détection des cibles pour l'atterrissage des drones. Nous présentons un cadre non supervisé robuste aux perturbations les plus courantes présentes dans les tâches de ce type.
  
  La deuxième partie explore les informations globales de couleur et de texture contenues dans les images. Nous présentons une analyse quantitative des différentes mesures de similarité existantes pour la mesure des distributions de couleur et d'énergie (signatures de texture) des images. Nous validons ces concepts dans un système de recherche d'images basé sur la similarité.
  
  La troisième et dernière partie de cette thèse aborde la relation entre les informations locales de couleur et la texture d'une image. Nous introduisons un cadre non supervisé pour obtenir des limites perceptuelles d'objets à partir de la décomposition spectrale d'une image. De plus, nous montrons une série de méthodes de segmentation à partir du groupe de caractéristiques calculées dans cet espace perceptif de couleur et de texture.
}

\enabstract{
  This thesis work deals with the extraction of features in images from low-level primitives for the understanding of scenes through the detection and segmentation of objects. Likewise, this work is divided into three parts.
  
  The first part explores the use of image contours in combination with some concepts of human perception such as the Helmholtz principle and the laws of Gestalt. These methods are then applied to the task of detecting targets for landing drones. We present an unsupervised framework robust to the most common disturbances present in tasks of this type.
  
  The second part explores the global color and texture information contained in the images. We present a quantitative analysis of the different existing similarity measures for the measurement of color and energy distributions (texture signatures) of images. We validate these concepts in a similarity-based image retrieval system.
  
  The third and last part of this thesis addresses the relationship between the local information of color and texture of an image. We introduce an unsupervised framework for obtaining perceptual boundaries of objects from the spectral decomposition of an image. In addition, we show a series of segmentation methods from the group of features calculated in this perceptual space of color and texture.
  
}

\frkeywords{Traitement d'image, Primitives de bas niveau, Détection, Segmentation, Compréhension de Scène, Apprentissage Automatique, Drone.}
\enkeywords{Image Processing, Low-level Primitives, Detection, Segmentation, Scene Understanding, Machine Learning, UAV.}

% Change this variable if you add or remove chapters
\newcommand*{\NumOfChapters}{7}

% Change this variable if you add or remove appendices
\newcommand*{\NumOfAppendices}{2}

% PDF metadata
\hypersetup{
	pdfauthor={\PhDname},
	pdfsubject={Manuscrit de thèse de doctorat},
	pdftitle={\PhDTitle}
}

\begin{document}
%\showthe\textwidth  % gives the width of the current document in pts. For this document width = 506.45908 (command usefull for matplotlib images)
	\pagenumbering{roman}
	
	\maketitle{}
	
    % created on 2019-12-13
% @author : bmazoyer
\newenvironment{dedication}
  {\clearpage           % we want a new page
   \thispagestyle{empty}% no header and footer
   \vspace*{\stretch{1}}% some space at the top 
   \itshape             % the text is in italics
   \raggedleft          % flush to the right margin
  }
  {\par % end the paragraph
   \vspace{\stretch{3}} % space at bottom is three times that at the top
   \clearpage           % finish off the page
  }
  
\begin{dedication}
Special dedication to all!!

\end{dedication}

%    \afterpage{\blankpage}
	\cleardoublepage
    
    % created on 09-04-2020
% @author : ebazan
\chapter*{Acknowledgements}
\addcontentsline{toc}{chapter}{Acknowledgments}

Thanks to the evaluation board for agreeing to review this manuscript.
    % \afterpage{\blankpage}
   	\cleardoublepage
    
    % created on 09-04-2020
% @author : ebazan
\chapter*{Abstract}
\addcontentsline{toc}{chapter}{Abstract}


\noindent This thesis work deals with the extraction of features in images from low-level primitives for the understanding of scenes through the detection and segmentation of objects. Likewise, this work is divided into three parts.
\newline 

\noindent The first part explores the use of image contours in combination with some concepts of human perception such as the Helmholtz principle and the laws of Gestalt. These methods are then applied to the task of detecting targets for landing drones. We present an unsupervised framework robust to the most common disturbances present in tasks of this type.
\newline

\noindent The second part explores the global color and texture information contained in the images. We present a quantitative analysis of the different existing similarity measures for the measurement of color and energy distributions (texture signatures) of images. We validate these concepts in a similarity-based image retrieval system.
\newline 

\noindent The third and last part of this thesis addresses the relationship between the local information of color and texture of an image. We introduce an unsupervised framework for obtaining perceptual boundaries of objects from the spectral decomposition of an image. In addition, we show a series of segmentation methods from the group of features calculated in this perceptual space of color and texture.

\vspace*{\fill}

\textbf{Keywords:} Image Processing, Low-level Primitives, Detection, Segmentation, Scene Understanding, Machine Learning, UAV.
%    \afterpage{\blankpage}
	\cleardoublepage
    
    % created on 09-04-2020
% @author : ebazan
\chapter*{Résumé}
\addcontentsline{toc}{chapter}{Résumé}

\noindent Ce travail de thèse traite de l'extraction de caractéristiques dans les images à partir de primitives de bas niveau pour la compréhension de scènes à partir de la détection et de la segmentation d'objets. Ce travail est divisé en trois parties.
\newline 

\noindent La première partie explore l'utilisation des contours d'image en combinaison avec certains concepts de la perception humaine tels que le principe de Helmholtz et les lois de la Gestalt. Ces méthodes sont ensuite appliquées à la tâche de détection des cibles pour l'atterrissage des drones. Nous présentons un cadre non supervisé robuste aux perturbations les plus courantes présentes dans les tâches de ce type.
\newline 

\noindent La deuxième partie explore les informations globales de couleur et de texture contenues dans les images. Nous présentons une analyse quantitative des différentes mesures de similarité existantes pour la mesure des distributions de couleur et d'énergie (signatures de texture) des images. Nous validons ces concepts dans un système de recherche d'images basé sur la similarité.
\newline 

\noindent La troisième et dernière partie de cette thèse aborde la relation entre les informations locales de couleur et la texture d'une image. Nous introduisons un cadre non supervisé pour obtenir des limites perceptuelles d'objets à partir de la décomposition spectrale d'une image. De plus, nous montrons une série de méthodes de segmentation à partir du groupe de caractéristiques calculées dans cet espace perceptif de couleur et de texture.
  
\vspace*{\fill}

\textbf{Mots clés:} Traitement d'image, Primitives de bas niveau, Détection, Segmentation, Compréhension de Scène, Apprentissage Automatique, Drone.
%    \afterpage{\blankpage}
	\cleardoublepage
    
    % created on 09-04-2020
% @author : ebazan
%\chapter*{}

\nomenclature{$\mathcal{F}\{\cdot\}$}{Fourier transform}

\nomenclature{$\mathcal{F^{-1}}\{\cdot\}$}{Inverse Fourier transform}

\nomenclature{$\sigma$}{Spread of 1-d Gaussian function}

\nomenclature{$\alpha$}{Sharpness (Gabor function major axis)}

\nomenclature{$\beta$}{Sharpness (Gabor function minor axis)}

\nomenclature{$\gamma$}{Sharpness of Gabor filter (major axis)}

\nomenclature{$\eta$}{Sharpness of Gabor filter (minor axis)}

\nomenclature{$\theta$}{Orientation angle of Gabor filter}

\nomenclature{$h(t)$}{1-d signal}

\nomenclature{$h(x,y)$}{2-d signal}

\nomenclature{$\phi$}{Phase shift Gabor filter}

\nomenclature{$g(t)$}{1-d Gabor filter in time domain}

\nomenclature{$g(t; f)$}{1-d Gabor filter in time domain at frequency $f$}

\nomenclature{$g(n)$}{Discrete 1-d Gabor filter in time domain}

\nomenclature{$g(x, y)$}{2-d Gabor filter in spatial domain}

\nomenclature{$g(x, y; f, \theta)$}{2-d Gabor filter in spatial domain t frequency $f$ and angle $\theta$}

\nomenclature{$g^{\ast}(t)$}{Complex conjugate of $g(t)$}

\nomenclature{$g(t){\ast}h(t)$}{Convolution of two 1-d functions in time}

\nomenclature{$\Delta$}{Uncertainty}

\nomenclature{$G(\upsilon)$}{1-d Gabor filter in frequency domain}

\nomenclature{$G(u, v)$}{2-d Gabor filter in frequency domain}

\nomenclature{$f$}{Frequency of Gabor function}

\nomenclature{$\upsilon$}{Frequency}

\nomenclature{$\omega$}{Radial frequency}

\nomenclature{$f_{0}$}{Frequency of Gabor function}

\nomenclature{$j$}{imaginary unit}

\nomenclature{$r(t; f)$}{Response of 1-d Gabor filter}

\nomenclature{$r(x, y; f, \theta)$}{Response of 2-d Gabor filter}

\nomenclature{$t$}{Time}

\nomenclature{$t_{0}$}{Location of Gabor function}

\nomenclature{$x$}{Spatial coordinate}

\nomenclature{$y$}{Spatial coordinate}

\nomenclature{$u$}{Frequency variable}

\nomenclature{$v$}{Frequency variable}

\renewcommand{\nomname}{Symbols and Abbreviations}
\printnomenclature

%    \afterpage{\blankpage}
	\cleardoublepage

    \tableofcontents
%    \listoffigures
%    \listoftables

%    \afterpage{\blankpage}
    \cleardoublepage

    \pagenumbering{arabic}   

%    \foreach \i in {1,2, ...,\NumOfChapters}{
%        % \input{chapters/\i}
%        \includefrom{chapters/\i/}{chapter\i}
%        %\afterpage{\blankpage}
%        \cleardoublepage
%    }
%    
%    \appendix
%    \foreach \i in {1,2,...,\NumOfAppendices}{
%        \includefrom{appendices/\i/}{appendix\i}
%        %\afterpage{\blankpage}
%        \cleardoublepage
%    }
	\includefrom{chapters/2/}{chapter2}
    %\afterpage{\blankpage}
    \cleardoublepage
    
    \bibliography{biblio}
    \pagebreak
   	
\end{document}