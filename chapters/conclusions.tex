% created on 20/07/2020
% @author : ericbzn
\phantomsection
\chapter*{Conclusion and Perspectives}\label{ch:general_conclusion}
\addcontentsline{toc}{chapter}{Conclusion}

\section*{Summary of Our Main Contributions}

\addcontentsline{toc}{section}{Summary of our Main Contributions}

This thesis deals with the study of low-level image information concerning human perception for scene understanding. In particular, we study intensity, color, and texture primitives. We validate our methodology on applications that present similar characteristics to the conditions encountered in drone vision-based tasks.

The algorithms proposed in this thesis managed to overcome some of the difficulties present in today's most widely used methods for image segmentation and object detection. Our algorithms do not need to rely on an a priori model, which is reflected in the independence of parameter definition and annotated databases. This methodology benefits the stages of a scene understanding system, which can be integrated into UAVs to develop vision-based tasks.

Throughout this thesis, we encounter different challenges linked to the nature of the application problems we seek to address. Specifically, one of the problems of vision-based application areas is real-time implementation. We did not explore this functionality in more detail; however, the proposed algorithms have elements that we can optimize; for example, the image convolution with filter bank using parallel computation or the calculation of optimal transport metric using a regularization technique. 

We further develop the conclusions of this work in the following list.

\begin{itemize}
	\item We present a framework for landing target detection, one of the main characteristics of visual tasks with drones. This detector uses the intensity information to obtain contours at different scales. The algorithm for detection operates in an unsupervised manner, reducing the number of correct operation parameters in different complex scenarios. In particular, our framework exploits the multi-scale image contours in a perceptual approach, taking into account the Helmholtz principle and the laws of organization of the Gestalt.
	
	\item We present a complete study of the color and texture properties present in images, and we evaluated the different options for representation and characterization of color and texture during the analysis of this information.
	
	\item We present two image retrieval systems, one based on color information and the other based on texture information. These systems served to test some concepts such as the optimal transport as a metric of similarity between distributions, the spectral analysis properties to represent an image's textures, and the importance of the color spaces.
	
	\item Motivated by the lack of a general analysis of the Gabor function optimized for the study of textures in images, we delved into the concepts of signal theory to propose a framework for the generation of a smooth Gabor filter bank: using Parseval's identity, we obtained a transfer function that is closest to one almost everywhere (in 1-d - frequency, and in 2-d - frequency and angle). That allows us to measure the energetic density spectrum accurately and use a true metric to measure the distance between two textures. Previous works on wavelet texture analysis are only approximative, either greyscale or color, and do not use a true distance. The most frequent measure is the Kullback–Leibler, which is a divergence.
	
	\item We present the complex multispectral decomposition of a natural image to analyze color, texture, and the relationship between this information in natural images. This decomposition results from a space-frequency study of Gabor filters and the study of color spaces of an image through its luminance and chrominance. Such research results in the Gabor-filter-based Complex Color (GCC) feature space that captures the interaction of perceptual color and texture information of an image.
	
	\item We present Gabor-filter-based Complex Color (GCC) feature space's utility, characteristics, and potential by implementing various unsupervised algorithms for natural image segmentation such as clustering algorithms (k-means, Gaussian Mixture, and Birch) and graph-based algorithms (Spectral clustering, MST threshold, and Normalized cuts). In addition to the application of these segmentation methods, we use the feature space to construct a series of high-level texture features, including fundamental frequency, dominant orientation, main texture-forming colors, among others. 

Furthermore, we showed that our methodology allows us to obtain the boundaries of the image in a perceptual way. We show that our methodology outperforms the BSDS benchmark score of state-of-the-art unsupervised methods for contour detection (Pb, Canny, Mean-shift, Felz-Hutt).
	
	\item All the frameworks presented in this document were implemented using open-access libraries in order to make them public. The algorithms presented were coded in python to use the different libraries and frameworks (OpenCV, NumPy, pandas, scikit-learn, scipy, etc.) existing in this language to work with images and public facts. This part of the thesis represents a significant programming effort hidden behind the results shown throughout this document.
	
	\item Finally, this thesis is part of the group of work that maintains traditional computer vision methods as the basis. Although nowadays it is possible to segment images with an accuracy close to that of a human using supervised algorithms and convolutional neural networks, we believe that it is possible to increase the performance, reliability, and explanation of such methods by combining them with systems based on physical phenomena of the vision. Even though AI solutions offer solutions with unprecedented accuracy scores their most criticised drawback today is their lack of explainability. We will comment on that in the Perspectives section below.	
	
\end{itemize}


\section*{Perspectives}
\addcontentsline{toc}{section}{Perspectives}

We can think of several promising perspectives both in terms of methodology and applications.

\begin{itemize}
	\item In chapter \ref{ch:landing_target_detection}, for the target detection system, there are different ways to improve the system. On the part of the methodology, it is possible to add more features of the image, such as the information of the color and the texture developed in part two of the thesis; This would make the system more robust, providing the possibility of creating markers with specific color and texture patterns. On the implementation side, this system can achieve the analysis and detection of targets in real-time by migrating the python code to some programming language, such as C ++, which allows the efficient parallelization of functions.
	
	 \item As for the image search systems presented in chapter \ref{ch:similarity_measures} it is possible to integrate both features (color and texture) in a single system that allows the search for natural color images. Moreover, in terms of implementation, it is possible to speed up the calculation time of the EMD by implementing a regularization of the measure.
	 
	 \item The Gabor filter we propose achieves a sense of optimality regarding the trade-off between space and frequency. However, this filter bank is non-orthogonal, i.e., the filter family may introduce redundant information. This feature does not affect our contour detection application as the EMD manages to handle the redundant information introduced mainly by the DC component of the signal.  If the objective is the perfect reconstruction of a signal keeping the space-frequency trade-off, a clue to follow is the study of the logarithmic Gabor function (log-Gabor) \citep{Field:OSA:1987}, which naturally eliminates the DC component by the logarithmic transformation of the Gabor domain \citep{Boukerroui.Noble.ea:JMIV:2004}.
	 
	 \item In chapter \ref{ch:perceptual_object_boundaries_detection}, we present a brief review of the state-of-the-art methods for superpixel computation. These methods obtain superpixels using intensity and (or) color information. We have a feature space (Gabor-filter-based) that represents the texture and color information of an image in which we can use a metric (EMD). It is natural then to think of an extension of the SLIC algorithm based on this space for the generation of texture superpixels. 
	 
	 \item In chapter \ref{ch:perceptual_object_boundaries_detection} we obtain the graph gradients for the luminance and chrominance channels of the complex color space. We can use these gradients in conjunction with the ground truth of the BSDS to learn (in a supervised manner) the weight of each color channel and see its perceptual importance in the segmentation task.
	 
	 \item Regarding the AI techniques for segmentation, we obtained results below the scores obtained with DL techniques; nevertheless, we do not use any model. A possibility is to use the proposed Gabor filter bank at the input of a DL network and obtain a model that will use perceptually relevant features. This network could be smaller, better regularized, and less greedy (trainable with less data).
	 
	 \item Using the feature space generated from the smooth filter bank over the complex color space and the optimal transport, we can use most of the morphological algorithms transparently on images containing color and texture; for example, a controlled watershed or MST on natural images. This clue has been explored on a superpixel basis; however, because of the computational time, the implementation on a pixel basis was not achieved. After optimization of the code, it might be possible to perform it on a pixel basis.

	 \item Finally, we consider that it is possible to generate a computational tool for the interactive segmentation of natural images. We can do it using the hierarchical watershed segmentation given by the perceptual boundaries obtained with the Gabor-filter Complex Color (GCC) detector presented in this thesis.
\end{itemize}


