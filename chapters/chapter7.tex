% created on 20/07/2020
% @author : ericbzn
\chapter{Conclusion and Perspectives}\label{ch:general_conclusion}

\section{Summary of our main contributions}

We mention the list of contributions of this thesis below.

\begin{itemize}
	\item We present a framework for landing target detection. The algorithm for detection operates in an unsupervised manner, reducing the number of parameters for correct operation in different complex scenarios, a main characteristic of visual tasks with drones. In particular, our framework exploits the information at different scales of the contours of the image in a perceptual approach, taking into account the Helmholtz principle and the laws of organization of the Gestalt.
	\item We present a complete study of the color and texture properties present in images. During the analysis of this information, we evaluated the different options for representation and characterization of color and texture.
	\item We present two image retrieval systems, one based on color information and the other based on texture information. These systems served to test some concepts such as the optimal transport as a metric of similarity between distributions; the properties of the spectral analysis for the representation of the textures of an image and the importance of the color spaces.
	\item Motivated by the lack of a general analysis of the Gabor function optimized for the study of textures in images, we delved into the concepts of signal theory to propose a framework for the generation of optimized and efficient Gabor filters for the study of textures.
	\item We present the complex multispectral decomposition of a natural image for the analysis of color, texture and the relationship between them in natural images. This decomposition is the result of the space-frequency study of the Gabor filters and the study of the color spaces of an image through its luminance and chrominance. Such research results in the Gabor complex color feature space that captures the interaction of perceptual color and texture information of an image.
	\item We present the utility, characteristics and potential of Gabor's complex color feature space by implementing various unsupervised algorithms for natural image segmentation. Through the BSDS image segmentation and contour detection benchmark, we show (quantitatively and quantitatively) that our methodology gives competitive results with state-of-the-art methods.
	\item All the frameworks presented in this document were implemented using open access libraries in order to make them public. The algorithms presented were coded in python in order to use the different libraries and frameworks (opencv, numpy, pandas, scikit-learn, scipy, etc.) existing in this language to work with images and public facts. This part of the thesis represents a significant programming effort that is hidden behind the results shown throughout this document.
	\item Finally, this thesis adds to the list of works that maintains traditional computer vision methods as a base. Although nowadays it is possible to segment images with an accuracy close to that of a human using supervised algorithms and convolutional neural networks, we believe that it is possible to increase the performance, reliability and explanatory of such methods by combining them with systems based on physical phenomena of the vision. We believe that today AI offers solutions but no answers, and we need both, especially in complex applications such as drone vision tasks. 
\end{itemize}


\section{Perspectives}
Several promising perspectives can be thought of, both in terms of methodology and applications.

\begin{itemize}
	\item In chapter \ref{ch:landing_target_detection}, for the target detection system, there are different ways to improve the system. On the part of the methodology, it is possible to add more features of the image, such as the information of the color and the texture developed in part two of the thesis; This would make the system more robust, providing the possibility of creating markers with specific color and texture patterns. On the implementation side, this system can achieve the analysis and detection of targets in real time by migrating the python code to some programming language, such as C ++, that allows the effective parallelization of functions.
	 \item As for the image search systems of chapter \ref{ch:similarity_measures}, it is possible to integrate both features (color and texture) in a single system that allows the search for natural color images. In terms of implementation, it is possible to speed up the calculation time of the EMD by implementing a regularization of the measure.
	 \item Regarding Gabor spectral analysis for texture analysis, a clue to follow is the logarithmic Gabor function (log-Gabor) recognized for solving the antialiasing problem naturally.
\end{itemize}

