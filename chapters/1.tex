% created on 30/03/2020
% @author : ebazan
\chapter{Introduction}
\section{Background and Motivation}
Computer Vision (CV) is about making machines capable of seeing and perceiving the world as humans do.

Replicating the behavior of human vision on machines is a difficult task. The tasks that human vision performs  quite effortlessly and effectively, for a machine could represent a complex task.

There have been a large number of studies to date which have tried to understad what is the nature of computation involved in visual tasks. Researchers from various fields of study such as biology, neuroscience and computer science have attempted to answer the question of how we might build the machines to be able to see and understand the world as we do.


\subsubsection{Computer Vision Milestone Works and History}
Computer vison is a rapidly evolving field. The three lines of research with which computer vision was born in the 1950s are replication of the eye, replication of the visual cortex, and replication of the rest of the brain. The joint work between psychologists and computer scientists led to the development of the Perceptron machine in 1957, the same year that marked the birth of the pixel.

By the 1960s, the problems of computer vision were linked to the representation of solid objects in 3D using simpler 2D figures. Such problems, along with other ideas such as connecting a camera to a computer to describe what he saw, motivated the creation of the MIT Artificial Intelligence Laboratory; After more than 50 years, this is an idea that continues to work.

Many of the vision algorithms that we know today, such as image edge extraction, line labialization, modeling, and object representation such as interconnections of smaller structures, optical flow, and motion estimation, were created in the 1970s. All of them driven by the first commercial applications of computer vision such as optical character recognition (OCR) and by the appearance of the first commercial camera.

The 80's can be characterized by the appearance of studies based on more rigorous mathematical analysis and on quantitative aspects. Many non-linear image analysis algorithms such as mathematical morphology were generalized for grayscale and image functions.

Investigation of 3D reconstructions from projections or scenes from multiple images as well as camera calibration came in the 1990s. Along with this, there were advances in the field of stereo imaging and stereo multi-view techniques. In addition, variations of graph cutting algorithms were used for image segmentation thus enabling high-level interpretation of the coherent regions of an image.

By the 2000s, there was a vast range of applications and fields related to computer vision. Some of the most relevant topics from that time are face detection, brought to fame by Viola and Jones, Lowe's Invariant Scale Features Transform (SIFT) developed as well as the appearance of Neural Networks, which gave way to learning methods. This decade marked a breakthrough for computer vision thanks to the large number of annotated data sets that were accessible on the internet. With them, it was possible to propose increasingly complex challenges, which allowed comparing and quantifying computer vision techniques.

Techniques based on learning methods have become popular in the last decade. A milestone of this time is the large ImageNet visual database, created in 2009 for research on visual object recognition software. Since 2010, the project has organized an annual software competition, the ImageNet Large Scale Visual Recognition Challenge (ILSVRC), where software competes to correctly classify objects and scenes in the database. Until 2012, no program had been as relevant as AlexNet, a deep convolutional neural network which surpassed the state of the art works of that time. The results of this competition undoubtedly marked the history of computer vision. In recent years, the study of CNNs has been exponential, giving way to novel systems such as GANs, which seek to make CNNs more robust.

Convolutionary neural networks not only became the choice of many researchers to carry out computer vision applications, large companies such as Google, Amazon, Facebook and Apple (GAFA) have promoted research and development in this field. A remarkable example is the DeepFace algorithm for facial recognition. The algorithm proposed by the Facebook researchers is capable of correctly identifying faces 97$\%$ of the time, this result being comparable to the detection that a human can do.

\subsubsection{Applications and Tasks}

The advancement of Computer Vision techniques has favored its use in a wide range of applications. The development has been outstanding in already traditional application areas such as multimedia, robotics or medicine. However, new areas of application continue to emerge such as augmented reality, automated driving, the Internet of Things and Industry 4.0, human-computer interaction and vision for the blind. Some less traditional areas where Computer Vision is increasingly present are astronomy, nanotechnology, new brain imaging techniques, among others.

While computer vision has outstripped the capabilities of human vision, computers have not completely replaced human personnel. For example, in the case of industrial vision systems tasks, say inspecting bottles or circuit boards on a production line, computer vision surpasses humans. However, in areas such as medical imaging, computer vision systems are only responsible for supplementing certain routine diagnoses that require a lot of time and experience from human doctors. This is largely related to the complexity of the task and the conditions under which the application is carried out. In the case of machine vision systems, working conditions are generally controlled, while in areas such as medicine, each patient image is different despite the fact that the acquisition system is the same.

Regardless of the application, computer vision systems must perform a number of tasks to achieve their end. Generally, these tasks include methods for acquiring, processing, analyzing, and understanding digital images, and extracting real-world data to produce symbolic information, for example, in the form of decisions. Depending on the context, understanding images can mean transforming visual images (the sensor input) into descriptions of the world that can interact with other processes and provoke appropriate action. This compression can be seen as the crumbling of the symbolic information in the image using geometric, physical, statistical or theory of learning models.

More particularly, the tasks of computer vision can be grouped into four more or less well-defined processing problems: Recognition, a classic problem of computer vision which is responsible for determining whether an image contains any object, characteristic or exercise. Some variants of this problem are the classification, identification and detection of objects from which many specialized tasks emerge, for example, content-based image search, pose estimation, optical character recognition, reading of 2D codes, facial recognition, shape recognition, among others.

Motion analysis, in which a sequence of images is processed to produce an estimate of the speed of one or more points of interest within a 3D image or scene. Some examples of this task are egomotion, object tracking, and optical flow.

The reconstruction of scenes, which is a task related to the computation of a 3D model from one or more images of a scene.

The restoration of images, whose objective is to remove those imperfections of an image generated by disturbances such as sensor noise or motion blur. Generally this task is carried out in the pre-processing of the image before passing it to a more complex mink algorithm. An example in which this task is applied is inpainting.

Today is an exciting time to work on computer vision. Thanks to free access databases on the internet, novel computer vision algorithms and increased computing power, machines can perform vision tasks faster and more efficiently than a human. In addition, access to these technologies has led to the development of novel applications, which favor not only the advancement of computer vision, but also favor society. Computer vision has become an integral technology in our daily life.

\section{Scope of the Thesis}

Unmanned aerial vehicles (UAVs) or drones are these flying machines that are increasingly present in our lives. The combination of different sensors of the inertial power station, the GPS and the flight computer allow the drones to remain stable in the air. However, the capabilities of a drone are extended when some type of visual sensor is integrated. Whether in entertainment activities used as toys, or in commercial activities used as work tools, drones allow us to see the world and obtain information from another perspective, literally.

However, a major limitation of these complex systems is the autonomy time. This parameter is mainly linked to the weight of the vehicle, given by the different components on board the aircraft and the capacity of the battery, which also influences the final weight of the equipment. In this sense, it is important to privilege those sensors that provide information from the outside world for navigation but also light and low energy consumption.

Taking this into account and remembering the areas of application and tasks of computer vision mentioned in the previous section, this thesis work will focus on those related to drones and object detection and scene comprehension tasks. Regarding the techniques and methods of the developed vision algorithms, this thesis exploits the now called traditional methods of computer vision, that is, those algorithms not based on neural network architectures and mainly on unsupervised methods.

\section{Problem Statement}

While drones are useful in a wide variety of applications, whether commercial, civil or military, in most cases an expert pilot is required to control the aircraft. The addition of external sensors to perceive the environment has been a recurring strategy that has made these aerial robots more manipulable, safer and even in some cases autonomous, that is, a drone is capable of performing a task without the need for a human intervention. For this, the drone must be able to move without getting lost but above all, it must be able to avoid potential obstacles on the way.

In general, we can consider that each drone application is a mission that involves three main moments: take-off, navigation, and landing. Commonly, the UAV  control in these phases is achieved with the use of conventional sensors, such as inertial sensors (IMUs) for orientation, and GPS for position. The drawback with the IMU is that suffers from bias error propagation due to the integral drift, while with the GPS signal is not always guaranteed. In urban or indoor environments, the satellite signal is low or unexisting. A recurrent technique to enhance the position accuracy implies the data fusion of pressure, ultrasonic, radars and laser range-finders sensors \cite{Tomic.Schmid.ea:IRAM:2012}. The fusion of data can provide the advantages of each sensor; however, the use of multiple sensors on board becomes expensive and impractical, taking into account the drone has a maximum payload capacity and the flight time depends on that. Contrariwise, visual sensors are passive, lightweight and can acquire valuable information about the surrounding structures, including color and textures, and UAV self-motion. 

Today one can use different visual sensors; such as monocular cameras \cite{Padhy.Xia.ea:TSC:2018}, stereo cameras \cite{Seitz.Curless.ea:CVPR:2006}, RGB-D cameras \cite{Huang.Bachrach.ea:RobR:2017}, fish-eye cameras \cite{Hrabar.Sukhatme:IROS:2004}, thermal \cite{Gaszczak.Breckon.ea:IRCV:2011}, among others. This wide range of sensors offers more options and flexibility to deal with the problems mentioned above. Based on the three moments in a UAV mission, we present a brief review of developed works to improve the controllability of a UAV. 

\subsection{UAV Navigation}
Vision-based techniques for UAV navigation are classified into two groups according to the intention and its consequence result in i) localization and mapping, ii) obstacle avoidance.

\subsubsection{Localization and Mapping}
The so-called Simultaneous Localization and Mapping (SLAM) is a technique that estimates the local pose of a robot and builds a 3D model of its surroundings employing visual sensors. The Visual Odometry (VO) \cite{Scaramuzza.Fraundorfer:RAM:2011} is responsible of the robot motion estimation while the maps are built with occupancy grid algorithms \cite{Thrun.Bu:AI:1996}. Taking into account the image information used to perform a SLAM, we can classify the methods in two classes.

\textbf{Feature-based Methods} extract a set of image features (e.g., lines, points) in a sequence of images. To do so, the invariant feature detectors most commonly used are Harris \cite{Harris.Stephens:AVC:1988}, SIFT \cite{Lowe:ICCV:1999}, FAST \cite{Rosten.Drummond:ECCV:2006}, SURF \cite{Bay.Ess.ea:CVIU:2008}. After that, the algorithm performs a feature matching through an invariant feature descriptor. The last stage is to perform the motion estimation using the data of other sensors; a local optimization is optional. 

The Parallel Tracking and Mapping (PTAM) \cite{Klein.Murray:ISMAR:2007} is one of the first and most used methods for SLAM. It is a feature-based algorithm that achieves robustness through tracking and mapping hundreds of features. It is the base for other approaches such as the Multi-Camera Parallel Tracking Mapping (MCPTAM) \cite{Harmat.Trentini.ea:IROS:2015} that uses multiple cameras to build a 3D map and calculate the robot position. \\

\textbf{Direct-Based Methods} make use of the image intensity information to estimate the structure and the motion of the robot. Unlike the feature-based methods, the direct methods compare the entire image between them to make a scene reconstruction. The use of image intensity information permits using different theoretical frameworks for SLAM. The DTAM method \cite{Newcombe.Lovegrove.ea:ICCV:2011} uses the minimization of a global, spatially-regularized energy functional while \cite{Moulon.Monasse.ea:ACCV:2012} uses a contrario framework to carry out a structure from motion.
Those methods provide more visual information about the environment giving a more meaningful representation to the human eye. 

Some of the first direct approaches \cite{Jin.Favaro.ea:VC:2003}, \cite{Molton.Davison.ea:BMVC:2004} treated salient feature patches as observations of locally planar regions on 3D world surfaces. This approach allows being robust to images where exits areas with textures and small gradients \cite{Lovegrove.Davison.ea:IVS:2011} or to blur images caused for camera-defocus \cite{Newcombe.Lovegrove.ea:ICCV:2011}.\\

The use of SLAM techniques for UAV navigation presents remarkable advantages. Feature-based methods can use a wide variety of feature detectors which counts typically with an optimization stage that allows having fast algorithms. Direct-based methods have the advantage to be robust to images degradations; they can lead better with images with texture and blurred zones; besides, the map produced is of an acceptable resolution. An interesting fact is that the strengths of the first group of methods are the weak points of the second and vice versa. A method that tries to gather the benefits of both approaches is the Semi-direct Visual Odometry \cite{Forster.Pizzoli.ea:ICRA:2014}; however, in general, the SLAM methods works in indoor environments, where the illumination conditions are static or controlled.\\

\subsubsection{Obstacle Avoidance}
An indispensable feature to increase the autonomy of the UAV navigation is the detection and avoidance of obstacles. This capability is of great importance for achieving free collisions missions in both, indoor and outdoor environments. A recurrent solution, as we early mentioned, is the multi-sensor data fusion. In \cite{Gageik.Benz.ea:ACCESS:2015} present a platform using low-cost ultrasound and IR sensors; however, despite the obtained results, it utilizes several sensors to recovers environment information and yet, it does not get a perceptual representation of the scene due to the low resolution and perceptive capacity of the sensors. On the other hand, vision-based techniques for obstacle avoidance could identify obstacles and in some cases classify the found object \cite{Li.Ye.ea:IROS:2016}. 

Visual methods for avoidance of obstacles can be classified into two groups. The first, SLAM-based techniques, make use of the principles described in the last subsection. The 3D reconstruction provides accurate and sophisticated maps and allows the air vehicle to travel with more information about the environment. In \cite{Moreno-Armendariz.Calvo:ICMEAE:2014}, takes this advantage to develop an obstacle avoidance approach for static and dynamic obstacles. 

The second group is the flow-based methods which historically, were inspired by the navigation of insects such as bees \cite{Srinivasan.Gregory:PTBS:1992} or flies \cite{Franceschini.Ruffier.ea:InTech:2012}. Many insects in the wild identify obstacles through the intensity of light. During the flight, their eyes produce an optical flow that provides accurate spatial information. Currently, there are also works inspired by the behavior of the human eye \cite{Al-Kaff.Meng.ea:IVS:2016}. The technique measures the object size from the idea that objects in the robot's field of vision are larger as the obstacle is closer.\\

The techniques for obstacle detection and avoidance present interesting characteristics and ideas; however, its implementation is strongly linked to an application under certain conditions. Their use would involve a recalibration or readjustment of parameters and, given the conditions in which a drone can operate, it is necessary to have more general and non-supervised methods.
 

\section{Objectives of the Thesis}

\section{General Literature Review}

\section{Organization of the Document}



KEY PHRASE: \textbf{Perceptual image information for object detection and segmentation}



