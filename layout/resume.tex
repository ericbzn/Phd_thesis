% created on 09-04-2020
% @author : ebazan
\chapter*{Résumé}
\addcontentsline{toc}{chapter}{Résumé}

\vspace*{-12ex}

\noindent Ce travail de thèse porte sur l'extraction de caractéristiques et de primitives de bas niveau à partir des informations perceptuelles de l'image pour comprendre des scènes. Motivés par les besoins et les problèmes de la navigation basée sur la vision des véhicules aériens sans pilote (UAV), nous proposons de nouvelles méthodes en nous concentrant sur les problèmes de compréhension de l'image. Ce travail explore trois informations principales dans une image: l'intensité, la couleur et la texture.
\newline 

\noindent Nous travaillons les informations d'intensité à travers les contours de l'image. Nous combinons ces informations avec des concepts issus de la perception humaine, tels que le principe de Helmholtz et les lois de la Gestalt, pour proposer un cadre non supervisé pour la détection et l'identification des objets. Nous validons cette méthodologie dans la dernière étape de la navigation par drone, juste avant l'atterrissage.
\newline 

\noindent Nous explorons les informations de couleur et de texture contenues dans les images. Tout d'abord, nous présentons une analyse de la couleur et de la texture en tant que distributions globales d'une image. Cette approche nous amène à étudier la théorie du transport optimal et ses propriétés comme véritable métrique de comparaison des distributions de couleur et de texture. Nous passons en revue et comparons les mesures de similarité les plus populaires entre les distributions pour montrer l'importance d'une métrique avec les propriétés correctes, telles que la non-négativité et la symétrie. Nous validons ces concepts dans deux systèmes de récupération d'images basés sur la similitude de la distribution des couleurs et de la distribution de l'énergie des textures.
\noindent Enfin, nous construisons une représentation d'image qui exploite la relation entre les informations de couleur et de texture. La représentation de l'image résulte de la décomposition spectrale de l'image, que l'on obtient par convolution avec une famille de filtres de Gabor. Nous présentons en détail les améliorations apportées au filtre Gabor et les propriétés des espaces colorimétriques complexes. Nous validons notre méthodologie avec une série d'algorithmes de détection des limites et de segmentation basés sur l'espace des caractéristiques perceptuelles calculé.

%\noindent La deuxième partie explore les informations globales de couleur et de texture contenues dans les images. Nous présentons une analyse quantitative des différentes mesures de similarité existantes pour la mesure des distributions de couleur et d'énergie (signatures de texture) des images. Nous validons ces concepts dans un système de recherche d'images basé sur la similarité.
%\newline 
%
%\noindent La troisième et dernière partie de cette thèse aborde la relation entre les informations locales de couleur et la texture d'une image. Nous introduisons un cadre non supervisé pour obtenir des limites perceptuelles d'objets à partir de la décomposition spectrale d'une image. De plus, nous montrons une série de méthodes de segmentation à partir du groupe de caractéristiques calculées dans cet espace perceptif de couleur et de texture.
  
\vspace*{\fill}

\textbf{Mots clés:} Primitives de Bas Niveau, Détection, Segmentation, Méthodes Non Supervisées, Compréhension de Scène, Apprentissage Automatique, Drone.