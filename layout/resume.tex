% created on 09-04-2020
% @author : ebazan
\chapter*{Résumé}
\addcontentsline{toc}{chapter}{Résumé}

\noindent Ce travail de thèse traite de l'extraction de caractéristiques dans les images à partir de primitives de bas niveau pour la compréhension de scènes à partir de la détection et de la segmentation d'objets. Ce travail est divisé en deux parties.
\newline 

\noindent La première partie explore l'utilisation des contours d'image en combinaison avec certains concepts de la perception humaine tels que le principe de Helmholtz et les lois de la Gestalt. Ces méthodes sont ensuite appliquées à la tâche de détection des cibles pour l'atterrissage des drones. Nous présentons un cadre non supervisé robuste aux perturbations les plus courantes présentes dans les tâches de ce type.
\newline 

\noindent La deuxième partie explore les informations globales de couleur et de texture contenues dans les images. Dans une première phase, nous présentons une analyse de la couleur et de la texture en tant que distributions globales d'une image. Cette approche conduit à étudier les différentes mesures de similitude entre les distributions existantes. Nous validons ces concepts dans deux systèmes de recherche d'images basés sur la similitude de la distribution des couleurs et de la distribution de l'énergie des textures. Dans la deuxième phase, nous abordons la relation entre la couleur et la texture localement dans une image, en reprenant certains concepts tels que la mesure de la similitude. Nous introduisons un cadre non supervisé pour obtenir des limites perceptuelles d'objets à partir de la décomposition spectrale d'une image. De plus, nous montrons une série de méthodes de segmentation à partir du groupe de caractéristiques calculées dans cet espace perceptif de couleur et de texture.


%\noindent La deuxième partie explore les informations globales de couleur et de texture contenues dans les images. Nous présentons une analyse quantitative des différentes mesures de similarité existantes pour la mesure des distributions de couleur et d'énergie (signatures de texture) des images. Nous validons ces concepts dans un système de recherche d'images basé sur la similarité.
%\newline 
%
%\noindent La troisième et dernière partie de cette thèse aborde la relation entre les informations locales de couleur et la texture d'une image. Nous introduisons un cadre non supervisé pour obtenir des limites perceptuelles d'objets à partir de la décomposition spectrale d'une image. De plus, nous montrons une série de méthodes de segmentation à partir du groupe de caractéristiques calculées dans cet espace perceptif de couleur et de texture.
  
\vspace*{\fill}

\textbf{Mots clés:} Traitement d'image, Primitives de bas niveau, Détection, Segmentation, Compréhension de Scène, Apprentissage Automatique, Drone.