% created on 09-04-2020
% @author : ebazan
\chapter*{Abstract}
\addcontentsline{toc}{chapter}{Abstract}


\noindent This thesis work deals with the extraction of features in images from low-level primitives for the understanding of scenes through the detection and segmentation of objects. Likewise, this work is divided into three parts.
\newline 

\noindent The first part explores the use of image contours in combination with some concepts of human perception such as the Helmholtz principle and the laws of Gestalt. These methods are then applied to the task of detecting targets for landing drones. We present an unsupervised framework robust to the most common disturbances present in tasks of this type.
\newline

\noindent The second part explores the global color and texture information contained in the images. We present a quantitative analysis of the different existing similarity measures for the measurement of color and energy distributions (texture signatures) of images. We validate these concepts in a similarity-based image retrieval system.
\newline 

\noindent The third and last part of this thesis addresses the relationship between the local information of color and texture of an image. We introduce an unsupervised framework for obtaining perceptual boundaries of objects from the spectral decomposition of an image. In addition, we show a series of segmentation methods from the group of features calculated in this perceptual space of color and texture.

\vspace*{\fill}

\textbf{Keywords:} Image Processing, Low-level Primitives, Detection, Segmentation, Scene Understanding, Machine Learning, UAV.